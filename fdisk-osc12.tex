%
% Copyright (C) 2012 Davidlohr Bueso <dave@gnu.org>
%
% Presentation for openSUSE conference 2012 - Prague.
% 
\documentclass{beamer}
\setbeamertemplate{navigation symbols}{}
\usepackage{beamerthemeshadow}

\begin{document}
\title{fdisk: a XXI Century Disk Partitioning Tool}  
\author{Davidlohr Bueso dave@gnu.org \\ Petr Uzel petr.uzel@suse.cz}
\date{\today} 

\begin{frame}
\titlepage
\end{frame}

\begin{frame}\frametitle{Table of contents}\tableofcontents
\end{frame} 

% INTRODUCTION
\section{Introduction} 
\begin{frame}\frametitle{Introduction}

\end{frame}
\subsection{Disk Partitioners}
\begin{frame}
\begin{itemize}
\item \textbf{fdisk}
\item parted
\item gdisk
\end{itemize}
\end{frame}


\section{Fdisk Problems} 
\subsection{Smelly, Legacy Code}

\begin{frame}\frametitle{Smelly, Legacy Code}
  \begin{columns}
    \begin{column}{.5\linewidth}
      The Linux fdisk program is over 20 years old and is a complex product of multiple authors, concepts, coding styles etc.\newline
      
      As a result, code is \textbf{glued} together and making it difficult and error prone to enhance an/or fix bugs.
    \end{column}
    \begin{column}{.5\linewidth}
      \includegraphics[scale=0.2]{img/glustick} 
    \end{column}
  \end{columns}
\end{frame}

\begin{frame}\frametitle{Smelly, Legacy Code}
\begin{figure}
\includegraphics[scale=0.4]{img/SmellyLegacyCode} 
\end{figure}
\end{frame}

\subsection{Stuck in the Past}
\begin{frame}\frametitle{Stuck in the Past}
  \begin{columns}
    \begin{column}{.5\linewidth}
      \includegraphics[scale=0.3]{img/linkman}
       \end{column}
    \begin{column}{.5\linewidth}
      \begin{itemize}
      \item DOS compatibility mode
      \item Only works with MBRs
      \item CHS addressing
      \item Mainframe style UIs\newline
      \end{itemize}
    \end{column}
  \end{columns}
\end{frame}

\subsection{Everyone Looses}
\begin{frame}\frametitle{Everyone Looses}
    \begin{columns}
      \begin{column}{.5\linewidth}
        \begin{block}{Hackers loose}
          Adding new code and extending functionality is difficult, tedious and error prone.
        \end{block}\pause
        \includegraphics[scale=0.3]{img/running-uphill}
      \end{column}
      \begin{column}{.5\linewidth} 
        \includegraphics[scale=0.5]{img/leaving}
        \begin{block}{Users loose}
        Fdisk \textbf{cannot} compete with other partitioning tools and thus looses users. Hey, healthy competition is good for everyone!
        \end{block}
      \end{column}
    \end{columns}
  \end{frame}

\section{Fixing Fdisk}
\begin{frame}\frametitle{Short Term}
  Short term goals:
  \begin{itemize}
  \item Cleanup and refactor current, legacy, code
  \item Create an internal API that abstracts disklabel concepts and specifications
  \item Add GUID Partition Table (GPT) support
  \end{itemize} 
\end{frame}

\begin{frame}\frametitle{Longer Term}
  Long term goals:
  \begin{itemize}
  \item Create an independant, libfdisk shared library.
  \item Rewrite cfdisk and sfdisk with new library.
  \end{itemize} 
\end{frame}

\section{GPT Overview}
\begin{frame}\frametitle{What is GPT?}
  A standard developed by Intel in the late '90s for the layout of the partition table on a physical hard disk.\newline

  It overcomes major limitations of MBRs and today forms part of the \textbf{UEFI} standard.
\end{frame}

\section{The Road Ahead}
\begin{frame}\frametitle{The Road Ahead}
  Still plenty of work to do, some things include:
 \begin{itemize}
  \item Enhance UIs ($libreadline$ - gdb style)
  \item Support more disklabels (APM, AIX)
  \item General cleanups and refactoring
  \item Documentation
  \item tests, tests, tests
  \end{itemize} 
\end{frame}

\begin{frame}\frametitle{Thank you}
  \center{ \includegraphics[scale=0.5]{img/thankyou} }
\end{frame}

\end{document}
